\chapter{导数与微分}  
\section{导数的有关定义}
\begin{definition}{导数}{derivative}
    设函数 $y=f(x),x\in D$。现有一点 $x_0\in D$且 $x_0+\Delta x \in D$。定义 $\Delta y=f(x_0+\Delta x)-f(x_0)$。

    若极限
    \begin{align}
        \lim_{\Delta x \to 0}\frac{\Delta y}{\Delta x}
    \end{align}
    存在,则称 $f(x)$ 在 $x_0$ 处可导,称该极限为 $f(x)$ 在 $x_0$ 处的导数。记作
    \begin{align}
        \lim_{\Delta x \to 0}\frac{\Delta y}{\Delta x} \triangleq f'(x_0)
    \end{align}
\end{definition}

\begin{note}
    \begin{enumerate}
        \item 等价定义:
        
        由于当 $x \to x_0$时,有 $f(x) \to f(x_0)$,而 $\Delta x =x-x_0$,$\Delta y = f(x)-f(x_0)$,从而导数又可以表示为
        \begin{align}
            \lim_{x \to x_0}\frac{ f(x)-f(x_0)}{x-x_0}\triangleq f'(x_0)
        \end{align}
        \item 左右导数
        
        $\Delta x \to 0$ 分为两种情况:$\Delta x  \to 0^{-}$ 以及
        $\Delta x  \to 0^{+}$;同理,$ x \to a$也分为两种情况:$ x \to a^{-}$以及 $ x \to a^{+}$。从而我们可以类似地得到:

        \begin{align}
            \lim_{\Delta x \to 0^{-}}\frac{\Delta y}{\Delta x} \triangleq f_{-}'(x_0) 
       \end{align}

       \begin{align}
           \lim_{\Delta x \to 0^{+}}\frac{\Delta y}{\Delta x} \triangleq f_{+}'(x_0) 
       \end{align}

        并且,可以给出一点导数存在的充要条件,也即:

        $f'(x_0)$ 存在 $\Leftrightarrow f_{-}'(x_0),f_{+}'(x_0) $ 均存在且相等。

        \item 可导性与连续性
        
        $f(x)$ 在 $x_0$ 处可导,则 $f(x)$ 在$x_0$ 处连续,反之则不然。证明从略。
        \begin{example}
            设 \[f(x) = \begin{cases}
                e^x-1, & x<0,\\
                \ln(1+2x), & x \ge 0
            \end{cases}\],
            
            求 $f'(0)$。
        \end{example}
        
        \begin{solution}
            先考虑连续性。
        
            由于 $f(0-0)=0=f(0+0)$,故 $f(x)$ 在 $x=0$ 处连续。
        
            再考虑可导性。
        
            由于 $f_{-}'(0)=\lim\limits_{x\to 0_{-}}\frac{f(x)-0}{x-0}=\lim\limits_{x \to 0_{-}}\frac{e^x-1}{x}=1$,
        
            $f_{+}'(0)=\lim\limits_{x\to 0_{+}}\frac{f(x)-0}{x-0}=\lim\limits_{x \to 0_{-}}\frac{\ln(1+2x)}{x}=2 \neq f_{-}'(0)$,
        
            故 $f'(0)$ 不存在。
        \end{solution}

        \item 已知 $f(x)$ 连续,
        
        若 $\lim\limits_{x \to a}\frac{f(x)-b}{x-a}=A$,则 $f(a)=b,f'(a)=A$。

        这是因为当 $\lim\limits_{x \to a}\frac{f(x)-b}{x-a}=A$时,由于分母为 $0$,则分子必须也为 $0$ 才能存在极限;根据导数定义又能推出 $f'(a)=A$。
        \end{enumerate}
    \section{可微}
    \begin{definition}{可微}{kewei}
    设函数 $y=f(x),x\in D$。现有一点 $x_0\in D$且 $x_0+\Delta x \in D$。定义 $\Delta y=f(x_0+\Delta x)-f(x_0)$。若$\Delta y=A\Delta x+o(\Delta x)$,则称 $f(x)$在 $x_0$ 可微。
    \end{definition}
        
\end{note}



